
\documentclass[12pt]{article}
\usepackage{graphicx}
\usepackage[margin=0.2in]{geometry}
\usepackage[utf8]{inputenc}

\usepackage{hyperref}

\title{Entwicklungsdoku}
%\author{
%        author
%}
\date{\today}

\begin{document}
	\maketitle
	
	%\begin{abstract}
	%This is the paper's abstract \ldots
	%\end{abstract}
	
	\section{Typgraph und Regeln in AGG}
	
%	\begin{figure}[ht]
%		\centering
%		\includegraphics[scale=1.0]{typgraph_fullNames.png}
%%		\vspace{-10pt}
%		\caption{(Typgraph in AGG}
%		\label{fig:typgraph_in_agg}
%	\end{figure}
%	
%		\begin{figure}[ht]
%		\centering
%		\includegraphics[scale=1.0]{executeNonLoop_fullNames.png}
%		%		\vspace{-10pt}
%		\caption{(Regel executeNonLoop in AGG}
%		\label{fig:rule_executeNonLoop}
%	\end{figure}
	
	
	\section{49 essential Critical Pairs}
	
	\textbf{Figure \ref{fig:result_1}:} Die Überlappung findet statt in: 2, 5, 8, 9, 10. Bei 2-5 und 8-9-10 handelt es sich jeweils um MCRs. Keine Überlappung an weiteren (boundardy)Knoten. $\Rightarrow$ minimal CP!
	
%	\begin{figure}[ht]
%		\centering
%		\includegraphics[scale=0.7]{1.png}
%		\vspace{-10pt}
%		\caption{(1) CP das minimal überlappt ist. Knoten des intialReason: 5:Cursor, 2:State, 8:Queue, 9:Element, 10:Element, Anzahl der MCR: 3}
%		\label{fig:result_1}
%	\end{figure}
%	
%	
%	\textbf{Figure \ref{fig:result_2}:} Die Überlappung findet statt in: 2, 5, 8, 9, 10. Bei 8-9-10 handelt es sich um ein MCR. Obowohl 2-5 einen MCR bilden könnten ist dies hier nicht der Fall. Dementsprechend sind hier beide Knoten ein boundary Knoten. Korrekt? $\Rightarrow$ kein minimal CP!
%	
%	\begin{figure}[ht]
%		\centering
%		\includegraphics[scale=0.7]{2.png}
%		\vspace{-10pt}
%		\caption{(2) essential CP  aber kein minimal CP}
%		\label{fig:result_2}
%	\end{figure}
%	
%	
%	\textbf{Figure \ref{fig:result_3}:} Die Überlappung findet statt in: 2, 5, 8, 9, 10.  Bei 8-9-10 handelt es sich um ein MCR. 2 und 5 sind jeweils boundary Knoten. $\Rightarrow$ kein minimal CP!
%	
%	\begin{figure}[ht]
%		\centering
%		\includegraphics[scale=0.7]{3.png}
%		\vspace{-10pt}
%		\caption{(3) essential CP  aber kein minimal CP}
%		\label{fig:result_3}
%	\end{figure}
%	
%	
%	\textbf{Figure \ref{fig:result_4}:} Die Überlappung findet statt in: 2, 5, 8, 9, 10. Bei 5-2 und 8-9 handelt es sich jeweils um MCRs. 10 ist ein boundary Knoten. $\Rightarrow$ kein minimal CP!
%	
%	\begin{figure}[ht]
%		\centering
%		\includegraphics[scale=0.7]{4.png}
%		\vspace{-10pt}
%		\caption{(4) essential CP  aber kein minimal CP}
%		\label{fig:result_4}
%	\end{figure}
%	
%	
%	\textbf{Figure \ref{fig:result_5}:} Die Überlappung findet statt in: 2, 5, 8, 9, 10. Bei 5-2 und 9-10 handelt es sich jeweils um MCRs. 8 ist ein boundary Knoten. $\Rightarrow$ kein minimal CP!
%	
%	\begin{figure}[ht]
%		\centering
%		\includegraphics[scale=0.7]{5.png}
%		\vspace{-10pt}
%		\caption{(5) essential CP  aber kein minimal CP}
%		\label{fig:result_5}
%	\end{figure}
%	
%	
%	\textbf{Figure \ref{fig:result_6}:} Die Überlappung findet statt in: 2, 5, 8, 9, 10. Bei 8-9 handelt es sich um ein MCR. 2, 5 und 10 sind boundary Knoten. $\Rightarrow$ kein minimal CP!
%	
%	\begin{figure}[ht]
%		\centering
%		\includegraphics[scale=0.7]{6.png}
%		\vspace{-10pt}
%		\caption{(6) essential CP  aber kein minimal CP}
%		\label{fig:result_6}
%	\end{figure}
%	
%	
%	\textbf{Figure \ref{fig:result_7}:} Die Überlappung findet statt in: 2, 5, 8, 9, 10. Bei 8-9 handelt es sich um ein MCR. 2, 5 und 10 sind boundary Knoten. $\Rightarrow$ kein minimal CP!
%	
%	\begin{figure}[ht]
%		\centering
%		\includegraphics[scale=0.7]{7.png}
%		\vspace{-10pt}
%		\caption{(7) essential CP  aber kein minimal CP}
%		\label{fig:result_7}
%	\end{figure}
%	
%	
%	\textbf{Figure \ref{fig:result_8}:} Die Überlappung findet statt in: 2, 5, 8, 9, 10. Bei 9-10 handelt es sich um ein MCR. 2, 5 und 8 sind boundary Knoten. $\Rightarrow$ kein minimal CP!
%	
%	\begin{figure}[ht]
%		\centering
%		\includegraphics[scale=0.7]{8.png}
%		\vspace{-10pt}
%		\caption{(8) essential CP  aber kein minimal CP}
%		\label{fig:result_8}
%	\end{figure}
%	
%	
%	\textbf{Figure \ref{fig:result_9}:} Die Überlappung findet statt in: 2, 5, 8, 9, 10. Bei 9-10 handelt es sich um ein MCR. 2, 5 und 8 sind boundary Knoten. $\Rightarrow$ kein minimal CP!
%	
%	\begin{figure}[ht]
%		\centering
%		\includegraphics[scale=0.7]{9.png}
%		\vspace{-10pt}
%		\caption{(9) essential CP  aber kein minimal CP}
%		\label{fig:result_9}
%	\end{figure}
%	
%	
%	\textbf{Figure \ref{fig:result_10}:} Die Überlappung findet statt in: 2, 5, 8, 9, 10. Bei 5-2 handelt es sich um ein MCR. 8, 9 und 10 sind boundary Knoten. $\Rightarrow$ kein minimal CP!
%	
%	\begin{figure}[ht]
%		\centering
%		\includegraphics[scale=0.7]{10.png}
%		\vspace{-10pt}
%		\caption{(10) essential CP  aber kein minimal CP}
%		\label{fig:result_10}
%	\end{figure}
%	
%	
%	\textbf{Figure \ref{fig:result_11}:} Die Überlappung findet statt in: 2, 5, 8, 9, 10. Bei 5-2 handelt es sich um ein MCR. 8, 9 und 10 sind boundary Knoten. $\Rightarrow$ kein minimal CP!
%	
%	\begin{figure}[ht]
%		\centering
%		\includegraphics[scale=0.7]{11.png}
%		\vspace{-10pt}
%		\caption{(11) essential CP  aber kein minimal CP}
%		\label{fig:result_11}
%	\end{figure}
%	
%	
%	\textbf{Figure \ref{fig:result_12}:} Die Überlappung findet statt in: 2, 8, 9, 10. Bei 8-9-10 handelt es sich um ein MCR. 2 ist ein boundary Knoten. $\Rightarrow$ kein minimal CP!
%	
%	\begin{figure}[ht]
%		\centering
%		\includegraphics[scale=0.7]{12.png}
%		\vspace{-10pt}
%		\caption{(12) essential CP  aber kein minimal CP}
%		\label{fig:result_12}
%	\end{figure}
%	
%	
%	\textbf{Figure \ref{fig:result_13}:} Die Überlappung findet statt in:  2, 8, 9, 10. Bei 8-9-10 handelt es sich um ein MCR. 2 ist ein boundary Knoten. $\Rightarrow$ kein minimal CP!
%	
%	\begin{figure}[ht]
%		\centering
%		\includegraphics[scale=0.7]{13.png}
%		\vspace{-10pt}
%		\caption{(13) essential CP  aber kein minimal CP}
%		\label{fig:result_13}
%	\end{figure}
%	
%	
%	\textbf{Figure \ref{fig:result_14}:} Die Überlappung findet statt in: 2, 8, 9, 10. Bei 8-9 handelt es sich um ein MCR. 2 und 10 sind boundary Knoten. $\Rightarrow$ kein minimal CP!
%	
%	\begin{figure}[ht]
%		\centering
%		\includegraphics[scale=0.7]{14.png}
%		\vspace{-10pt}
%		\caption{(14) essential CP  aber kein minimal CP}
%		\label{fig:result_14}
%	\end{figure}
%	
%	
%	\textbf{Figure \ref{fig:result_15}:} Die Überlappung findet statt in: 2, 8, 9, 10. Bei 8-9 handelt es sich um ein MCR. 2 und 10 sind boundary Knoten. $\Rightarrow$ kein minimal CP!
%	
%	\begin{figure}[ht]
%		\centering
%		\includegraphics[scale=0.7]{15.png}
%		\vspace{-10pt}
%		\caption{(15) essential CP  aber kein minimal CP}
%		\label{fig:result_15}
%	\end{figure}
%	
%	
%	\textbf{Figure \ref{fig:result_16}:} Die Überlappung findet statt in: 2, 8, 9, 10. Bei 9-10 handelt es sich um ein MCR. 2 und 8 sind boundary Knoten. $\Rightarrow$ kein minimal CP!
%	
%	\begin{figure}[ht]
%		\centering
%		\includegraphics[scale=0.7]{16.png}
%		\vspace{-10pt}
%		\caption{(16) essential CP  aber kein minimal CP}
%		\label{fig:result_16}
%	\end{figure}
%	
%	
%	\textbf{Figure \ref{fig:result_17}:} Die Überlappung findet statt in: 2, 8, 9, 10. Bei 9-10 handelt es sich um ein MCR. 2 und 8 sind boundary Knoten. $\Rightarrow$ kein minimal CP!
%	
%	\begin{figure}[ht]
%		\centering
%		\includegraphics[scale=0.7]{17.png}
%		\vspace{-10pt}
%		\caption{(17) essential CP  aber kein minimal CP}
%		\label{fig:result_17}
%	\end{figure}
%	
%	
%	\textbf{Figure \ref{fig:result_18}:} Die Überlappung findet statt in: 5, 8, 9, 10. Bei 8-9-10 handelt es sich um ein MCR. 5 ist ein boundary Knoten. $\Rightarrow$ kein minimal CP!
%	
%	\begin{figure}[ht]
%		\centering
%		\includegraphics[scale=0.7]{18.png}
%		\vspace{-10pt}
%		\caption{(18) essential CP  aber kein minimal CP}
%		\label{fig:result_18}
%	\end{figure}
%	
%	
%	\textbf{Figure \ref{fig:result_19}:} Die Überlappung findet statt in: 5, 8, 9, 10. Bei 8-9 handelt es sich um ein MCR. 5 und 10 sind boundary Knoten. $\Rightarrow$ kein minimal CP!
%	
%	\begin{figure}[ht]
%		\centering
%		\includegraphics[scale=0.7]{19.png}
%		\vspace{-10pt}
%		\caption{(19) essential CP  aber kein minimal CP}
%		\label{fig:result_19}
%	\end{figure}
%	
%	
%	\textbf{Figure \ref{fig:result_20}:} Die Überlappung findet statt in: 5, 8, 9, 10. Bei 9-10 handelt es sich um ein MCR. 5 und 8 sind boundary Knoten. $\Rightarrow$ kein minimal CP!
%	
%	\begin{figure}[ht]
%		\centering
%		\includegraphics[scale=0.7]{20.png}
%		\vspace{-10pt}
%		\caption{(20) essential CP  aber kein minimal CP}
%		\label{fig:result_20}
%	\end{figure}
%	
%	
%	\textbf{Figure \ref{fig:result_21}:} Die Überlappung findet statt in: 2, 5, 8, 10. Bei 5-2 handelt es sich ein MCR. 8 und 10 sind boundary Knoten. $\Rightarrow$ kein minimal CP!
%	
%	\begin{figure}[ht]
%		\centering
%		\includegraphics[scale=0.7]{21.png}
%		\vspace{-10pt}
%		\caption{(21) essential CP  aber kein minimal CP}
%		\label{fig:result_21}
%	\end{figure}
%	
%	
%	\textbf{Figure \ref{fig:result_22}:} Die Überlappung findet statt in: 2, 5, 8, 10. Bei 5-2 handelt es sich ein MCR. 8 und 10 sind boundary Knoten. $\Rightarrow$ kein minimal CP!
%	
%	\begin{figure}[ht]
%		\centering
%		\includegraphics[scale=0.7]{22.png}
%		\vspace{-10pt}
%		\caption{(22) essential CP  aber kein minimal CP}
%		\label{fig:result_22}
%	\end{figure}
%	
%	
%	\textbf{Figure \ref{fig:result_23}:} Die Überlappung findet statt in: 2, 5, 8, 9. Bei 5-2 und 8-9 handelt es sich um MCRs. Es gibt keine boundary Knoten. $\Rightarrow$ minimal CP
%	
%	\begin{figure}[ht]
%		\centering
%		\includegraphics[scale=0.7]{23.png}
%		\vspace{-10pt}
%		\caption{(23) CP das minimal überlappt ist. Knoten des intialReason: 5:Cursor, 2:State, 8:Queue, 9:Element, Anzahl der MCR: 2}
%		\label{fig:result_23}
%	\end{figure}
%	
%	
%	\textbf{Figure \ref{fig:result_24}:} Die Überlappung findet statt in: 2, 5, 8, 9. Bei 8-9 handelt es sich um ein MCR. 2 und 5 sind boundary Knoten. $\Rightarrow$ kein minimal CP
%	
%	\begin{figure}[ht]
%		\centering
%		\includegraphics[scale=0.7]{24.png}
%		\vspace{-10pt}
%		\caption{(24) essential CP  aber kein minimal CP}
%		\label{fig:result_24}
%	\end{figure}
%	
%	
%	\textbf{Figure \ref{fig:result_25}:} Die Überlappung findet statt in: 2, 5, 8, 9. Bei 8-9 handelt es sich um ein MCR. 2 und 5 sind boundary Knoten. $\Rightarrow$ kein minimal CP
%	
%	\begin{figure}[ht]
%		\centering
%		\includegraphics[scale=0.7]{25.png}
%		\vspace{-10pt}
%		\caption{(25) essential CP  aber kein minimal CP}
%		\label{fig:result_25}
%	\end{figure}
%	
%	
%	\textbf{Figure \ref{fig:result_26}:} Die Überlappung findet statt in: 2, 5, 8, 9. Bei 5-2 handelt es sich um ein MCR. 8 und 9 sind boundary Knoten. $\Rightarrow$ kein minimal CP
%	
%	\begin{figure}[ht]
%		\centering
%		\includegraphics[scale=0.7]{26.png}
%		\vspace{-10pt}
%		\caption{(26) essential CP  aber kein minimal CP}
%		\label{fig:result_26}
%	\end{figure}
%	
%	
%	\textbf{Figure \ref{fig:result_27}:} Die Überlappung findet statt in: 2, 5, 8, 9. Bei 5-2 handelt es sich um ein MCR. 8 und 9 sind boundary Knoten. $\Rightarrow$ kein minimal CP
%	
%	\begin{figure}[ht]
%		\centering
%		\includegraphics[scale=0.7]{27.png}
%		\vspace{-10pt}
%		\caption{(27) essential CP  aber kein minimal CP}
%		\label{fig:result_27}
%	\end{figure}
%	
%	
%	\textbf{Figure \ref{fig:result_28}:} Die Überlappung findet statt in: 2, 5, 9, 10. Bei 5-2 und 9-10 handelt es sich um MCRs. Es gibt keine boundary Knoten. $\Rightarrow$ minimal CP
%	
%	\begin{figure}[ht]
%		\centering
%		\includegraphics[scale=0.7]{28.png}
%		\vspace{-10pt}
%		\caption{(28) CP das minimal überlappt ist. Knoten des intialReason: 5:Cursor, 2:State, 9:Element, 10:Element, Anzahl der MCR: 2}
%		\label{fig:result_28}
%	\end{figure}
%	
%	
%	\textbf{Figure \ref{fig:result_29}:} Die Überlappung findet statt in: 2, 5, 9, 10. Bei 9-10 handelt es sich um ein MCR. 2 und 5 sind boundary Knoten. $\Rightarrow$ kein minimal CP
%	
%	\begin{figure}[ht]
%		\centering
%		\includegraphics[scale=0.7]{29.png}
%		\vspace{-10pt}
%		\caption{(29) essential CP  aber kein minimal CP}
%		\label{fig:result_29}
%	\end{figure}
%	
%	
%	\textbf{Figure \ref{fig:result_30}:} Die Überlappung findet statt in: 2, 5, 9, 10. Bei 9-10 handelt es sich um ein MCR. 2 und 5 sind boundary Knoten. $\Rightarrow$ kein minimal CP
%	
%	\begin{figure}[ht]
%		\centering
%		\includegraphics[scale=0.7]{30.png}
%		\vspace{-10pt}
%		\caption{(30) essential CP  aber kein minimal CP}
%		\label{fig:result_30}
%	\end{figure}
%	
%	
%	\textbf{Figure \ref{fig:result_31}:} Die Überlappung findet statt in: 2, 5, 9, 10. Bei 5-2 handelt es sich um ein MCR. 9 und 10 sind boundary Knoten. $\Rightarrow$ kein minimal CP
%	
%	\begin{figure}[ht]
%		\centering
%		\includegraphics[scale=0.7]{31.png}
%		\vspace{-10pt}
%		\caption{(31) essential CP  aber kein minimal CP}
%		\label{fig:result_31}
%	\end{figure}
%	
%	
%	\textbf{Figure \ref{fig:result_32}:} Die Überlappung findet statt in: 2, 5, 9, 10. Bei 5-2 handelt es sich um ein MCR. 9 und 10 sind boundary Knoten. $\Rightarrow$ kein minimal CP
%	
%	\begin{figure}[ht]
%		\centering
%		\includegraphics[scale=0.7]{32.png}
%		\vspace{-10pt}
%		\caption{(32) essential CP  aber kein minimal CP}
%		\label{fig:result_32}
%	\end{figure}
%	
%	
%	\textbf{Figure \ref{fig:result_33}:} Die Überlappung findet statt in: 8, 9, 10. Bei 8-9-10 handelt es sich um ein MCR. Es gibt keine boundary Knoten. $\Rightarrow$ minimal CP
%	
%	\begin{figure}[ht]
%		\centering
%		\includegraphics[scale=0.7]{33.png}
%		\vspace{-10pt}
%		\caption{(33) CP das minimal überlappt ist. Knoten des intialReason: 8:Queue, 9:Element, 10:Element, Anzahl der MCR: 2}
%		\label{fig:result_33}
%	\end{figure}
%	
%	
%	\textbf{Figure \ref{fig:result_34}:} Die Überlappung findet statt in: 8, 9, 10. Bei 8-9 handelt es sich um ein MCR. 8 ist ein boundary Knoten. $\Rightarrow$ kein minimal CP
%	
%	\begin{figure}[ht]
%		\centering
%		\includegraphics[scale=0.7]{34.png}
%		\vspace{-10pt}
%		\caption{(34) essential CP  aber kein minimal CP}
%		\label{fig:result_34}
%	\end{figure}
%
%
%	\textbf{Figure \ref{fig:result_35}:} Die Überlappung findet statt in: 8, 9, 10. Bei 9-10 handelt es sich um ein MCR. 8 ist ein boundary Knoten. $\Rightarrow$ kein minimal CP
%
%	\begin{figure}[ht]
%		\centering
%		\includegraphics[scale=0.7]{35.png}
%		\vspace{-10pt}
%		\caption{(35) essential CP  aber kein minimal CP}
%		\label{fig:result_35}
%	\end{figure}
%
%
%	\textbf{Figure \ref{fig:result_36}:} Die Überlappung findet statt in: 2, 8, 9. Bei 8-9 handelt es sich um ein MCR. 2 ist ein boundary Knoten. $\Rightarrow$ kein minimal CP
%	
%	\begin{figure}[ht]
%		\centering
%		\includegraphics[scale=0.7]{36.png}
%		\vspace{-10pt}
%		\caption{(36) essential CP  aber kein minimal CP}
%		\label{fig:result_36}
%	\end{figure}
%
%
%	\textbf{Figure \ref{fig:result_37}:} Die Überlappung findet statt in: 2, 8, 9. Bei 8-9 handelt es sich um ein MCR. 2 ist ein boundary Knoten. $\Rightarrow$ kein minimal CP
%
%	\begin{figure}[ht]
%		\centering
%		\includegraphics[scale=0.7]{37.png}
%		\vspace{-10pt}
%		\caption{(37) essential CP  aber kein minimal CP}
%		\label{fig:result_37}
%	\end{figure}
%
%
%	\textbf{Figure \ref{fig:result_38}:} Die Überlappung findet statt in: 2, 9, 10. Bei 0-10 handelt es sich um ein MCR. 2 ist ein boundary Knoten. $\Rightarrow$ kein minimal CP
%
%	\begin{figure}[ht]
%		\centering
%		\includegraphics[scale=0.7]{38.png}
%		\vspace{-10pt}
%		\caption{(38) essential CP  aber kein minimal CP}
%		\label{fig:result_38}
%	\end{figure}
%
%
%	\textbf{Figure \ref{fig:result_39}:} Die Überlappung findet statt in: 5, 9, 10. Bei 9-10 handelt es sich um ein MCR. 2 ist ein boundary Knoten. $\Rightarrow$ kein minimal CP
%
%	\begin{figure}[ht]
%		\centering
%		\includegraphics[scale=0.7]{39.png}
%		\vspace{-10pt}
%		\caption{(39) essential CP  aber kein minimal CP}
%		\label{fig:result_39}
%	\end{figure}
%
%
%	\textbf{Figure \ref{fig:result_40}:} Die Überlappung findet statt in: 5, 8, 9. Bei 8-9 handelt es sich um ein MCR. 5 ist ein boundary Knoten. $\Rightarrow$ kein minimal CP
%
%	\begin{figure}[ht]
%		\centering
%		\includegraphics[scale=0.7]{40.png}
%		\vspace{-10pt}
%		\caption{(40) essential CP  aber kein minimal CP}
%		\label{fig:result_40}
%	\end{figure}
%
%	
%	\textbf{Figure \ref{fig:result_41}:} Die Überlappung findet statt in: 5, 9, 10. Bei 9-10 handelt es sich um ein MCR. 5 ist ein boundary Knoten. $\Rightarrow$ kein minimal CP
%	
%	\begin{figure}[ht]
%		\centering
%		\includegraphics[scale=0.7]{41.png}
%		\vspace{-10pt}
%		\caption{(41) essential CP  aber kein minimal CP}
%		\label{fig:result_41}
%	\end{figure}
%
%
%	\textbf{Figure \ref{fig:result_42}:} Die Überlappung findet statt in: 2, 5, 8. Bei 5-2 handelt es sich um ein MCR. 8 ist ein boundary Knoten. $\Rightarrow$ kein minimal CP
%
%	\begin{figure}[ht]
%		\centering
%		\includegraphics[scale=0.7]{42.png}
%		\vspace{-10pt}
%		\caption{(42) essential CP  aber kein minimal CP}
%		\label{fig:result_42}
%	\end{figure}
%
%
%	\textbf{Figure \ref{fig:result_43}:} Die Überlappung findet statt in: 2, 5, 10. Bei 5-2 handelt es sich um ein MCR. 10 ist ein boundary Knoten. $\Rightarrow$ kein minimal CP
%	
%	\begin{figure}[ht]
%		\centering
%		\includegraphics[scale=0.7]{43.png}
%		\vspace{-10pt}
%		\caption{(43) essential CP  aber kein minimal CP}
%		\label{fig:result_43}
%	\end{figure}
%
%
%	\textbf{Figure \ref{fig:result_44}:} Die Überlappung findet statt in: 2, 5, 10. Bei 5-2 handelt es sich um ein MCR. 10 ist ein boundary Knoten. $\Rightarrow$ kein minimal CP
%	
%	\begin{figure}[ht]
%		\centering
%		\includegraphics[scale=0.7]{44.png}
%		\vspace{-10pt}
%		\caption{(44) essential CP  aber kein minimal CP}
%		\label{fig:result_44}
%	\end{figure}
%
%
%	\textbf{Figure \ref{fig:result_45}:} Die Überlappung findet statt in: 2, 5, 9. Bei 5-2 handelt es sich um ein MCR. 9 ist ein boundary Knoten. $\Rightarrow$ kein minimal CP
%	
%	\begin{figure}[ht]
%		\centering
%		\includegraphics[scale=0.7]{45.png}
%		\vspace{-10pt}
%		\caption{(45) essential CP  aber kein minimal CP}
%		\label{fig:result_45}
%	\end{figure}
%
%
%	\textbf{Figure \ref{fig:result_46}:} Die Überlappung findet statt in: 2, 5, 9. Bei 5-2 handelt es sich um ein MCR. 9 ist ein boundary Knoten. $\Rightarrow$ kein minimal CP
%	
%	\begin{figure}[ht]
%		\centering
%		\includegraphics[scale=0.7]{46.png}
%		\vspace{-10pt}
%		\caption{(46) essential CP  aber kein minimal CP}
%		\label{fig:result_46}
%	\end{figure}
%
%
%	\textbf{Figure \ref{fig:result_47}:} Die Überlappung findet statt in: 8, 9. Bei 8-9 handelt es sich um ein MCR. Es gibt keine boundary Knoten. $\Rightarrow$ minimal CP
%	
%	\begin{figure}[ht]
%		\centering
%		\includegraphics[scale=0.7]{47.png}
%		\vspace{-10pt}
%		\caption{(47) CP das minimal überlappt ist. Knoten des intialReason: 8:Queue, 9:Element, Anzahl der MCR: 1}
%		\label{fig:result_47}
%	\end{figure}
%
%
%	\textbf{Figure \ref{fig:result_48}:} Die Überlappung findet statt in: 9, 10. Bei 9-10 handelt es sich um ein MCR. Es gibt keine boundary Knoten. $\Rightarrow$ minimal CP
%	
%	\begin{figure}[ht]
%		\centering
%		\includegraphics[scale=0.7]{48.png}
%		\vspace{-10pt}
%		\caption{(48) CP das minimal überlappt ist. Knoten des intialReason: 9:Element, 10:Element, Anzahl der MCR: 1}
%		\label{fig:result_48}
%	\end{figure}
%
%
%	\textbf{Figure \ref{fig:result_49}:} Die Überlappung findet statt in: 2, 5. Bei 5-2 handelt es sich um ein MCR. Es gibt keine boundary Knoten. $\Rightarrow$ minimal CP
%	
%	\begin{figure}[ht]
%		\centering
%		\includegraphics[scale=0.7]{49.png}
%		\vspace{-10pt}
%		\caption{(49) CP das minimal überlappt ist. Knoten des intialReason: 5:Cursor, 2:State, Anzahl der MCR: 1}
%		\label{fig:result_49}
%	\end{figure}

	\subsection{Zusammenfassung der Ergebnisse:}
	Von den 49 essential CPs sind 7 minimal CPs (Ergebnis 1, 23, 28, 33, 47, 48, 49).
\end{document}